\markdownRendererHeadingOne{Instruções para textos}\markdownRendererInterblockSeparator
{}\markdownRendererInputFencedCode{./_markdown_main/3f4c7f30381bf80f5304b5282ff2318d.verbatim}{md}\markdownRendererInterblockSeparator
{}\markdownRendererStrongEmphasis{Esse testo estará negrito}, \markdownRendererEmphasis{já este estará itálico} e ``este estará entre aspas\markdownRendererFootnote{Este é um exemplo de nota de rodapé.}''.\markdownRendererInterblockSeparator
{}Uma lista não enumerada:\markdownRendererInterblockSeparator
{}\markdownRendererUlBeginTight
\markdownRendererUlItem Olá\markdownRendererUlItemEnd 
\markdownRendererUlItem Mundo\markdownRendererUlItemEnd 
\markdownRendererUlEndTight \markdownRendererInterblockSeparator
{}Uma lista enumerada:\markdownRendererInterblockSeparator
{}\markdownRendererOlBeginTight
\markdownRendererOlItemWithNumber{1}Olá\markdownRendererOlItemEnd 
\markdownRendererOlItemWithNumber{2}Mundo.\markdownRendererOlItemEnd 
\markdownRendererOlEndTight \markdownRendererInterblockSeparator
{}\markdownRendererHeadingOne{Instuções para equações}\markdownRendererInterblockSeparator
{}\markdownRendererInputFencedCode{./_markdown_main/b5db2c8f4f48ccdf8d7a1d1a32441117.verbatim}{md}\markdownRendererInterblockSeparator
{}Exemplo de equação na linha: $E = mc^2$.\markdownRendererInterblockSeparator
{}Exemplo de equação sem numeração: $$ E = mc^2 $$\markdownRendererInterblockSeparator
{}\markdownRendererHeadingOne{Instuções para fazer referências}\markdownRendererInterblockSeparator
{}\markdownRendererInputFencedCode{./_markdown_main/a9bda170eef7d5ada69124a48cad91c7.verbatim}{md}\markdownRendererInterblockSeparator
{}Exemplo de referência \markdownRendererCite{1}+{}{}{moyses1}.\markdownRendererInterblockSeparator
{}Exemplo de citação:\markdownRendererInterblockSeparator
{}\markdownRendererBlockQuoteBegin
Lex II: Mutationes motis proportionalem esse vi motrici impressæ, et fieri secundum lineam retam qua vis illa imprimitur. - N. Isaac \markdownRendererCite{1}+{}{}{principia1}
\markdownRendererBlockQuoteEnd \markdownRendererInterblockSeparator
{}\markdownRendererHeadingOne{Instuções para inserir imagens}\markdownRendererInterblockSeparator
{}\markdownRendererInputFencedCode{./_markdown_main/49d43c6a0234a55cb7cb5c235de1ad52.verbatim}{md}\markdownRendererInterblockSeparator
{}Veja abaixo a \autoref{fig:exemplo}.\markdownRendererInterblockSeparator
{}\markdownRendererImage{exemplo}{img/NomeDoArquivo.png}{img/NomeDoArquivo.png}{Título da imagem}\markdownRendererInterblockSeparator
{}\markdownRendererHeadingOne{Instuções para inserir códigos}\markdownRendererInterblockSeparator
{}\markdownRendererInputFencedCode{./_markdown_main/7d9db1c8b95d9d619e8913042967596a.verbatim}{md}\markdownRendererInterblockSeparator
{}Olá Mundo em c:\markdownRendererInterblockSeparator
{}\markdownRendererInputFencedCode{./_markdown_main/2e1a7f15aadb08ce11db4779088aa3fa.verbatim}{c}\markdownRendererInterblockSeparator
{}Os estilos suportados são:\markdownRendererInterblockSeparator
{}\markdownRendererUlBeginTight
\markdownRendererUlItem \markdownRendererCodeSpan{c}: Para \markdownRendererCodeSpan{c}\markdownRendererUlItemEnd 
\markdownRendererUlItem \markdownRendererCodeSpan{f90}: Para Fortran 90\markdownRendererUlItemEnd 
\markdownRendererUlEndTight \relax