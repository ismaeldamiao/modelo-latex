\documentclass[
   % -- opções da classe memoir --
   12pt,                         % tamanho da fonte.
   a4paper,                      % tamanho do papel.
   % -- opções do pacote babel --
   spanish,                      % idioma adicional para hifenização.
   brazil,                       % O último idioma é o principal do documento.
   xcolor=table                  % Tabelas coloridadas.
]{abntex2}
% ---
% PACOTES
% ---
\usepackage{amssymb}                   % Para exibir símbolos de conjuntos de números (reais, etc...).
\usepackage{amsmath}                   % Para adcionar equações
\usepackage{amsfonts}                  % Fontes para notação matemática.
\usepackage{amsthm}                    % Teoremas
\usepackage{xcolor}                    % Para colorir
\usepackage{thmtools}                  % Front end para amsthm (\declaretheorem)
\usepackage[T1]{fontenc}               % Selecao de codigos de fonte.
\usepackage[utf8x]{inputenc}           % Codificacao do documento (conversão automática dos acentos).
\usepackage{ucs}                       % Complemento do anterior.
\usepackage{indentfirst}               % Indenta o primeiro parágrafo de cada seção.
\usepackage{color}                     % Controle das cores.
\usepackage{microtype}                 % para melhorias de justificação.
\usepackage{lipsum}                    % para geração de dummy text.
\usepackage{fancyhdr}                  % Pemite alterações no cabeçalho e rodapé.
\usepackage{hyperref}                  % Cria formatação automática de PDF.
\usepackage[hyphenbreaks]{breakurl}    % Quebra de linha em url.
\usepackage{geometry}                  % Permite configurar as margens da página.
\usepackage[num,overcite]{abntex2cite} % Citações padrão ABNT, em ordem alfabética.
\citebrackets[]                        % Citações com coxetes.
\usepackage{setspace}                  % Espaço duplo entre parágrafos.
\usepackage{bibentry}                  % Para mostrar artigos publicados.
% --- 
% CONFIGURAÇÕES DE PACOTES
% --- 

\hypersetup{
   pdftitle={CV I.F.F. dos Santos},
   colorlinks=true,   % false: boxed links; true: colored links
   linkcolor=blue,    % color of internal links
   citecolor=blue,    % color of links to bibliography
   filecolor=magenta, % color of file links
   urlcolor=blue
}

\begin{document}
\nobibliography{works}
% ------------------------------------------------------------------------------
% ELEMENTOS TEXTUAIS
% ------------------------------------------------------------------------------
   \begin{center}
      \textbf{ISMAEL FELIPE FERREIRA DOS SANTOS}
   \end{center}
   \begin{flushleft}
      \small
      Endereço: Rua 2ª Travessa da Esperança, Jacintinho, Maceió-AL.\\
      Telefone e WhatsApp: \href{tel:+5582993452161}{(82) 99345-2161}\\
      E-mail: \href{mailto:ismaellxd@gmail.com}{ismaellxd@gmail.com}\\
      Lattes: \href{http://lattes.cnpq.br/1281887099263383}{http://lattes.cnpq.br/1281887099263383}
   \end{flushleft}

% ------------------
% Escolaridade
% ------------------
   \begin{flushleft}
      \hline \vspace{2pt}
      \textbf{Escolaridade} \vspace{2pt} \hline
   \end{flushleft}
   \begin{itemize}[nosep]
      \item[ ] \textbf{Bacharelado em Física:}
      \begin{itemize}[nosep]
        \item[ ] Universidade Federal de Alagoas – UFAL. Desde 2018 (6º período).
      \end{itemize}
      \item[ ] \textbf{Ensino médio:}
      \begin{itemize}[nosep]
        \item[ ] E. E. Profº Theonilo Gama. Concluído em março de 2018.
      \end{itemize}
   \end{itemize}

% ------------------
% Idioma
% ------------------
   \begin{flushleft}
      \hline \vspace{2pt}
      \textbf{Idioma} \vspace{2pt} \hline
   \end{flushleft}
   \begin{itemize}[nosep]
      \item[ ] \textbf{Espanhol:}
      \begin{itemize}[nosep]
        \item[ ] Lê bem, fala bem e escreve bem.
      \end{itemize}
      \item[ ] \textbf{Inglês:}
      \begin{itemize}[nosep]
        \item[ ] Lê razoavelmente, escreve razoavelmente.
      \end{itemize}
   \end{itemize}

% ------------------
% Softwares de domínio
% ------------------
   \begin{flushleft}
      \hline \vspace{2pt}
      \textbf{Softwares de domínio} \vspace{2pt} \hline
   \end{flushleft}
   \begin{itemize}[nosep]
      \item Linux;
      \item Linguagens de programação:
      \begin{itemize}[nosep]
         \item BASH;
         \item C;
         \item Fortran 90.
      \end{itemize}
      \item \LaTeX;
      \item LibreOffice;
      \item GNUplot;
      \item xmGrace.
   \end{itemize}

% ------------------
% Experiências
% ------------------
   \begin{flushleft}
      \hline \vspace{2pt}
      \textbf{Experiências} \vspace{2pt} \hline
   \end{flushleft}
   \begin{itemize}[nosep]
      \item Monitor nas edições de 2018 e 2019 da expofísica da UFAL.
      \item Membro do Grupo de Trabalho para discussão e criação do Estatuto do
      Diretório Acadêmico do Instituto de Física, órgão de representação estudantil.
      \item Dois períodos como monitor no laboratório de Física Experimental 1.
      \item Dois anos de Iniciação Científica na área de
      transporte em sistemas de baixa dimensionalidade.
   \end{itemize}

% ------------------
% Publicações
% ------------------
   \begin{flushleft}
      \hline \vspace{2pt}
      \textbf{Publicações} \vspace{2pt} \hline
   \end{flushleft}
   \begin{enumerate}[nosep]
      \item \bibentry{DOSSANTOS2020125126}
      \item \bibentry{doi:10.1142/S0129183121500406}
      \item \bibentry{DOSSANTOS2021125773}
   \end{enumerate}

\end{document}
